\documentclass[USenglish]{article}	
% for 2-column layout use \documentclass[USenglish,twocolumn]{article}

\usepackage[utf8]{inputenc}				%(only for the pdftex engine)
%\RequirePackage[no-math]{fontspec}[2017/03/31]%(only for the luatex or the xetex engine)

\usepackage{lmodern} 
\usepackage{microtype}
\usepackage[numbers,square,sort&compress]{natbib}
\usepackage[big,online]{dgruyter}	%values: small,big | online,print,work

% New theorem-like environments will be introduced by using the commands \theoremstyle and \newtheorem.
% Please note that the environments proof and definition are already defined within dgryuter.sty.
\theoremstyle{dgthm}
\newtheorem{theorem}{Theorem}
\newtheorem{corollary}{Corollary}
\newtheorem{proposition}{Proposition}
\newtheorem{lemma}{Lemma}
\newtheorem{assertion}{Assertion}
\newtheorem{result}{Result}
\newtheorem{conclusion}{Conclusion}

\theoremstyle{dgdef}
\newtheorem{definition}{Definition}
\newtheorem{example}{Example}
\newtheorem{remark}{Remark}

\usepackage{Sweave}
\begin{document}
\input{manuscript_JQAS-concordance}

	
%%%--------------------------------------------%%%
	\articletype{Research Article}
	\received{Month	DD, YYYY}
	\revised{Month	DD, YYYY}
  \accepted{Month	DD, YYYY}
  \journalname{De~Gruyter~Journal}
  \journalyear{YYYY}
  \journalvolume{XX}
  \journalissue{X}
  \startpage{1}
  \aop
  \DOI{10.1515/sample-YYYY-XXXX}
%%%--------------------------------------------%%%

\title{Insert your title here}
\runningtitle{Short title}
%\subtitle{Insert subtitle if needed}

\author*[1]{Mena C.R. Whalen}
%\ use * to mark the author as the corresponding author
\author[2]{Brian Mills}
\author[3]{Gregory J. Matthews} 
\runningauthor{Whalen et al.}
\affil[1]{\protect\raggedright 
Loyola University Chicago, Department of Mathematics and Statistics, Chicago, IL, USA, e-mail: mwhalen3@luc.edu}
\affil[2]{\protect\raggedright 
University of Texas at Austin, Department of Kinesiology and Health Education, Austin, TX, USA, e-mail: brian.mills@austin.utexas.edu}
\affil[3]{\protect\raggedright 
Loyola University Chicago, Department of Mathematics and Statistics, Chicago, IL, USA, e-mail: gmatthews1@luc.edu}
	
	
%\communicated{...}
%\dedication{...}
	
\abstract{}

\keywords{Baseball, Change Point Analysis}

\maketitle
	
	
\section{Introduction} 

The first professional baseball team in the United States, the Cincinnati Red Stockings, was formed in 1869 (\cite{BBHOF1869}).   Many leagues came and went in the late 1800s, but National League (NL), formed in 1876, emerged as the predominant league of the time.  A few decades later, the American League (AL) began growing in popularity and eventually reached an agreement with the NL to be the two major leagues of baseball with the winner of each league playing in the World Series starting in 1903.  

Throughout the history of baseball in the United States, the game has gone through many changes and distinct eras.  For example, the time period between approximately 1900-1919 is often referred to as the "Dead Ball Era" and was marked by low scoring games and dominant pitching.  Another more recent example would be the "Steroid Era" which lasted from approximately 1994 through 2005 and was characterized by a rapid increase in power hitting largely attributed to players using performance enhancing drugs. 

More specifically, \cite{Woltring2018} mentions six eras of modern baseball": 
"Baseball has endured much change over the course of its history, and because of constant change, the modern era of baseball has been segmented into six distinct sub-eras. A common list presented at Baseball-Reference described the eras as the Dead Ball Era (1901-1919), the Live Ball Era (1920-1941), the Integration Era (1942-1960), the Expansion Era (1961-1976), the Free Agency Era (1977-1993) and the Long Ball/Steroid Era (1994-2005)."  \cite{Woltring2018} notes that they name a seventh era after 2006, which they term the Post Steroid Era.  

While many baseball writers have attempted to define the different eras of baseball, there has also been some academic work that has sought to empirically define eras in baseball.  \cite{Groothius2017} looked for structural breaks in univariate times series of performance measures over the period from 1871-2020.  They analyzed four statistics: slugging percentage (SLG, home run (HR) rate, batting average (BA), and runs batted in (RBI) rate.   For each of these statistics, the computed the mean and standard deviation (SD) across all players who had at least 100 at bats in a given season to yield a univariate time series for each of these statistics.  They then used the Lagrange Multiplier (LM) unit root test proposed in \cite{LeeandStrazicich2003} to find structural breaks.  They identified structural breaks in slugging percentage in 1921 and 1992, the first of which marks the end of the Dead Ball Era and the latter corresponding to the start of the steroid era. 

\cite{LeeFort2005} looks for structural changes in competitive balance of the two league American and National.  Use methods from Andrews1993, Bai1997, 1999 and Bai and Perron 1998 and 2003 to look for break points between 1901 -1999.  They measure competitive balance in two ways: 1) \cite{Noll1988} and \cite{Scully1989} and 2) Lee 2004. They find break point sin competitive balance in 1912, 1926, and 1933 for the NL and in 1926 and 1957 in the AL.  

Baseball is not the only sport where this type of analysis has been applied.  \cite{PalaciosHuerta2004} looked for structural changes in soccer using data from British soccer leagues through 1996.  They had data from division I (Premier League) and II starting in the late 1800s, and data from lower divisions III and IV from just after WWII starting in 1947.  They identify a number of change points.  Notably they identify a change point in the mean of margin of victory in 1925 related to the change in the definition of offsides (changed from 3 players to 2 players), an change points in the variability of number of goals in the early 1980s and 1992 corresponding to the change in number of points for a win (i.e. went from 2 points to 3 points) and a change in the backpass rule, respectively.  \cite{FortLee2007} looked for structural breaks in major North American sports other than baseball (i.e. basketball (NBA), hockey (NHL), and American football (NFL)).  They identified a number of change points related to competitive balance in each sport that often, though not always, correspond to league expansion, league mergers, or other major events in a sport (e.g. increased number of foreign players in the NBA in the late 1990s/early 2000s).

% # WHAT DID THEY FIND?   ADD MORE HERE.  

All of the previous work mentioned here focuses on change point analysis in {\bf univariate} time series. However, recent methodolgical developments in change point analysis allow for change point analysis in {\bf multivariate} time series, which is the focus of this current manuscript.  This work uses the Double CUSUM Binary Segmentation algorithm (\cite{Cho2016,ChoFryzlewicz2014}) to look for change points in Major League Baseball at two main levels.  First, we look for structural changes in the league as a whole across teams to empirically define different eras in baseball.  Second, we look for change points within a team to determine eras of a team.  This second analysis can be used to identify the beginning and end of so called "dynasties", periods of sustained excellent performance by a team.  In addition, we can also identify the opposite, sustained periods of poor performance.  

% @LeeFort2008: Attendance and the Uncertainty-of-Outcome Hypothesis in Baseball.  Identify Break points in attendance in 1918 and 1945 for both leagues.  For AL only: 1963, 1978, 1994.  For NL only: 1976.  Using Bai and Perron From table 3 in Mills and Fort 2014
% @MillsFort2014: LEAGUE-LEVEL ATTENDANCE AND OUTCOME UNCERTAINTY IN U.S. PRO SPORTS LEAGUES.  Looks at NHL, NFL, NBA.  Rottenburg 1956 looked at baseball.  Identify break points in NBA, NFL, and NHL.  Table 3
% @MillsSalaga2015: NCAA Basketball: League balance using stats.  
% @MillsFort2018: team-level: Attendence.  
% @SalagaFort2017: College football


% <!-- Just Notes:  -->
% <!-- Palacios-Huerta I. (2004). Structural changes during a Century of the World’s most popular sport. Statistical Methods and Applications, 13, 241–258 -->
% <!-- @PalaciosHuerta2004 Structural Breaks in soccer -->
% 
% 
% <!-- @Nieswiadomy2012 Was there a structural break in Barry Bonds’s Bat?  Looking for structural breaks in Bonds OPS times series. Use the uni root test on monthly data from 86-07.  The identify break points in June 1993 and Sept 2000.  -->
% 
% 
% <!-- In contrast to their methods, we use  more modern methods Double CUSUM Binary Segmentation algorithm (@Cho2016) and the Sparsified Binary Segmentation algorithm (@ChoFryzlewicz2014). -->
% 
% <!-- https://onlinelibrary.wiley.com/doi/epdf/10.1111/j.1465-7295.2007.00026.x -->
% <!-- Fort and Lee 2007: Looks for structural breaks in NBA, NHL, and NFL -->
% 
% 
% 
% <!-- https://www.baseball-reference.com/bullpen/History_of_baseball_in_the_United_States -->
% 
% <!-- https://thesportjournal.org/article/examining-perceptions-of-baseballs-eras/ -->

%SOME MORE STUFF


% <!-- https://pubmed.ncbi.nlm.nih.gov/30747582/? -->
% 
% 
% <!-- The goal of this work is to use multivariate change point analysis to identify change points in MLB as a whole to objectively identify eras in the games history, but also to look for change points to identify team specific eras.   -->
% 
% <!-- The work that is the most closely related to our own is that of @Groothius2017, which looked for structural breaks (Note: the terms "structural breaks" and "change points" are used interchangeably in this manuscript) in univariate times series.   -->
% 
% <!-- univariate time series of the means and SD's of several statistics (i.e. SLUG, HR rate, RBI rate, and batting average) to look for structural breaks via the Lagrange Multiplier (LM) unit root test @LeeandStrazicich2003.  Notably they find structural breaks in slugging percentage in 1921 and 1992.  In contrast to their methods, we use  more modern methods Double CUSUM Binary Segmentation algorithm (@Cho2016) and the Sparsified Binary Segmentation algorithm (@ChoFryzlewicz2014). -->
% 
% 
% <!-- https://journals.sagepub.com/doi/10.1177/152700250100200204 -->
% 
% <!-- Deadball era: https://sabr.org/journal/article/the-rise-and-fall-of-the-deadball-era/ -->
% 
% <!-- http://mlb.mlb.com/mlb/history/mlb_history_people.jsp?story=com -->
% 
% <!-- https://www.nytimes.com/2011/09/25/sports/baseball/scoring-in-baseball-returns-to-dead-ball-levels.html?_r=1& -->
% 
% <!-- https://bleacherreport.com/articles/1676509-the-evolution-of-the-baseball-from-the-dead-ball-era-through-today -->
% 
% <!-- @FortLee2006:  -->
% <!-- Fort R., Lee Y. H. (2006). Stationarity and major league baseball attendance analysis. Journal of Sports Economics, 7, 408–415. -->
% 
% <!-- Fort R., Lee Y. H. (2007). Structural change, competitive balance, and the rest of the major leagues. Economic Inquiry, 45, 519–532. -->
% 
% <!-- Lee Y. H., Fort R. (2005). Structural change in baseball’s competitive balance: The great depression, team location, and racial integration. Economic Inquiry, 43, 158–169. -->
% 
% 
% <!-- @Nieswiadomy2012 Was there a structural break in Barry Bonds’s Bat?  Looking for structural breaks in Bonds OPS times series. Use the uni root test on monthly data from 86-07.  The identify break points in June 1993 and Sept 2000.   -->
% 
% <!-- degruyter.com/document/doi/10.1515/1559-0410.1305/html -->
% 
% <!-- Palacios-Huerta I. (2004). Structural changes during a Century of the World’s most popular sport. Statistical Methods and Applications, 13, 241–258 -->
% <!-- @PalaciosHuerta2004 Structural Breaks in soccer -->
% 
% <!-- @Groothius2017 is the most closely related work to ours.  They use univariate time series of the means and SD's of several statistics (i.e. SLUG, HR rate, RBI rate, and batting average) to look for structural breaks via the Lagrane Multiplier (LM) unit root test @LeeandStrazicich2003.  Notably they find structural breaks in slugging percentage in 1921 and 1992.  In contrast to their methods, we use  more modern methods Double CUSUM Binary Segmentation algorithm (@Cho2016) and the Sparsified Binary Segmentation algorithm (@ChoFryzlewicz2014).   -->
% 
% <!-- Fort, Rodney and Young Hoon Lee (2006). “Stationarity and Major League Baseball Attendance Analysis.” Journal of Sports Economics, 7 (4), 408–415.  -->
% 
% <!-- Fort, Rodney and Young Hoon Lee (2007). “Structural Change, Competitive Balance, and the Rest of the Major Leagues.” Economic Inquiry, 45 (3), 519-532.  -->
% 
% <!-- Lee, Young Hoon, and Rodney Fort (2005). “Structural Change in Baseball’s Competitive Balance: The Great Depression, Team Location, and Racial Integration.” Economic Inquiry, 43, 158-169.  -->
% 
% <!-- Sommers, P. (2008). “The Changing Hitting Performance Profile in Major League Baseball, 1966-2006.” Journal of Sports Economics, 9, 435-440.  -->
% 
% 
% 
% 
% 
% 
% 
% <!-- @Berry1999bridging they talk about bridging eras.   -->
% 
% <!-- When did the steroids era start: https://www.espn.com/mlb/topics/_/page/the-steroids-era#:~:text=Unlike%20other%20MLB%20%22eras%2C%22,leaguewide%20PED%20testing%20until%202003. -->
% 
% <!-- Traditional wisdom: -->
% <!-- 1900-1919 dead ball era -->
% 
% <!-- From Woltring: -->
% <!-- "Baseball has endured much change over the course of its history, and because of constant change, the modern era of baseball has been segmented into six distinct sub-eras. A common list presented at Baseball-Reference described the eras as the Dead Ball Era (1901-1919), the Live Ball Era (1920-1941), the Integration Era (1942-1960), the Expansion Era (1961-1976), the Free Agency Era (1977-1993) and the Long Ball/Steroid Era (1994-2005) (17). This study runs through the 2011 season and a seventh era will be added and labeled the Post Steroid Era (2006-2011)" -->
% 
% 
% <!-- https://www.baseball-reference.com/bullpen/Deadball_Era -->
% <!-- 1901-1920 -->
% 
% <!-- Mound was lowered in december 1968.   -->
% <!-- https://www.baseball-reference.com/bullpen/Pitcher%27s_mound -->
% 
% <!-- https://www.espn.com/mlb/story/_/id/33238595/major-league-baseball-stops-testing-players-steroids-nearly-20-years-report-says -->
% 
% <!-- https://www.semanticscholar.org/paper/Was-There-a-Structural-Break-in-Barry-Bonds's-Bat-Nieswiadomy-Strazicich/0a32effa10a3d26ea4675efcca3f13229efe0f7d -->
% 

\section{Data}
%EXPLAIN DATA NEEDS FOR 1) ERAS and 2) DYNASTIES.
We obtained multi-year baseball statistics from the Lahman database (\cite{Friendly2021Lahman}). Over the course of the sport's history, numerous statistics have been collected, and for our analysis, we focused on year-end statistics pertaining to the teams. In order to implement the DC algorithm, each time series is required to be the same length.  As a result, when studying eras in our analysis we focus on a set of 16 teams that have all existed over the period from 1900 to 2020 because this period of time covers most of the history of organized professional baseball and there are enough teams that have all existed over this time period to perform meaningful analysis.  If we had instead chosen to extend this period to, for instance, 1890 to 2020, we would only have had XX teams available.   This timeframe of $T=120$ years encompassed a total of 16 franchise teams listed in table \ref{tab:120year:franch}. The statistics of interest for these teams include runs, hits, home runs, balls, strikeouts, at-bats, stolen bases, number of games played in a season, attendance, runs against, hits against, home runs against, balls against, and strikeouts against. However, a few statistics, such as balls, stolen bases, strikeouts, and attendance, had missing data for certain years. To address this, we employed multiple imputations using classification and regression trees (\cite{Vanbuuren1999flexible}) to fill in the missing values and ensure consistent data across all analyses.  (HOW DID WE DO THRE IMPUTATION?  Was is single imputation?  )

For the dynasty analysis we used teams over different periods of time for every year that each team has existed....MORE DETAILS.  

\begin{table}
\begin{tabular}{|r|}
Franchise  \\
\hline
NYY Full names 
 NYY, BOS, LAD, ATL, CHW, CHC, CIN, CLE, DET, BAL, SFG, OAK, PHI, PIT, STL, and MIN\\
\hline
\end{tabular}
\caption{}
\label{tab:120year:franch}
\end{table}



\section{Methods}
%@ChoFryzlewicz2014 and @Cho2016

R Pacakge: \cite{hdbinseg}

A commonly employed method for detecting changes over time involves the CUSUM statistic in binary segmentation (CITE XX), which creates a tree diagram, dividing the time series after successfully identifying a change point until no further change points are detected. The magnitude of the CUSUM difference between two segments generally indicates a potential change point, depending on the assumptions and statistical tests applied to the time series. While this process works well for a single time series, detecting change points across a panel of multiple time series requires a more comprehensive approach. \cite{Cho2016} introduced the double CUSUM (DC) algorithm, assuming there are $n$ time series, all of the same length $T$, with unknown common change point locations. This algorithm utilizes the CUSUM values of multiple time series $(j = 1, \dots, n)$ to cross-sectionally compare time segments for potential change points.

To achieve this, the DC statistic from \cite{Cho2016} involves ordered CUSUM values for a given time location, favoring larger values as potential change point candidates. It incorporates a partial thresholding of the smaller ordered CUSUM values, effectively representing a scaled average of the most prominent values at a given location across all $n$ (WHY WAS "n" in quotes here?) time series. Maximizing the DC statistic across all time series and potential time locations results in a test statistic for detecting a change point. This test statistic is then compared against a specific testing criterion, denoted as $\pi_{n,T}^\phi$, leading to the partitioning of time series and the formation of a tree structure based on the detected change points.

To determine the testing criterion $\pi_{n,T}^\phi$, a Generalized Dynamic Factor Model (GDFM) bootstrapping algorithm is employed (CITE XX), accounting for potential correlations within and between high-dimensional time series. This methodology enables the detection of second-order change points, utilizing methodology from \cite{ChoFryzlewicz2014} Haar wavelet periodograms and cross-periodograms, as opposed to the traditional CUSUM statistic. For further details, refer to \cite{Cho2016} and \cite{ChoFryzlewicz2014}.  These methods were implemented using the \texttt{hdbinseg} \cite{hdbinseg} in the R programming language (\cite{R2023language}).

Discuss somewhere how we can find mean change points and ALSO variance change points.  

SEGUE parageraph Using eras and dynasties:::::::::

Since we are interested in finding where in time these different eras of baseball have occurred, we first examine each baseball statistic of interest as their own analysis for all 16 teams. Those baseball statistics are home runs, strikeouts, balls, stolen bases, runs, and attendance which were all standardized to the number of games played in a season since we are interested in seeing how the different statistics change from season to season. Using the methodology from \cite{ChoFryzlewicz2014} a thresholding parameter was determined from bootstrapping and then used to find where change points in mean and variance existed in the panel of teams data of a given baseball statistic. To examine the idea of changes throughout teams and types of statistics (initially), for each season the average baseball statistic was found from all the teams that year and then an average time series was created of each of the types of statistics of interest. Then the change point methodology was performed on those 6 time series of average statistic types.

Continuing our investigation, we explored the concept of "dynasties" within modern baseball teams, irrespective of their length of a teams existence. A dynasty, in this context, refers to a period characterized by significant success or potentially hardships. Such a dynasty is identified when a team experiences a noticeable change across all their statistics in a given season, whether it be in a positive or negative direction. This approach aligns well with our methodology, as it involves examining panels of time series data encompassing all statistics for a singular team.
The statistics of interest for this analysis include runs, hits, home runs, balls, strikeouts, runs against, hits against, home runs against, balls against, and strikeouts against. To ensure comparability, these statistics were standardized using the season average and standard deviation derived from all teams present in each respective season. By standardizing the statistics, we capture a team's offensive and defensive performance. Notably, if a team exhibits a high and positive offensive performance, it can be balanced to zero by large and negative defensive statistics when standardized. Therefore, if a team excels in both aspects of the game, it would have numerous time points above zero, whereas underperforming teams would display many below-zero values.For each team, we examined their ten statistics, analyzing both mean and variance changes to identify potential change points. By scrutinizing these indicators, we aimed to uncover shifts in a team's performance and assess their significance within the context of dynasties.

%hdbinseg: Change-Point Analysis of High-Dimensional Time Series via Binary Segmentation



\section{Results}
\subsection{Eras}
Blah blah 
\subsection{Dynasties}
Blah blah 

\begin{table}
\begin{tabular}{|r|r|}
Franchise & Change Point Year(s) \\
\hline
ANA       & 1977             \\
ATL       & 1900, 1990       \\
BAL       & 1959, 1999      \\
BOS       & 1918, 1937       \\
CHC       & 1891             \\
CHW       & 1970             \\
CIN       & 1918, 1952       \\
CLE       & 1944, 1959, 1993 \\
DET       & 1958, 1988       \\
FLA       & 2011             \\
HOU       & 1999             \\
KCR       & 1997             \\
LAD       & 1940, 1961       \\
MIL       & 1977, 1996       \\
MIN       & 1959             \\
PHI       & 1917, 1949       \\
PIT       & 1939             \\
SDP       & 1977             \\
SEA       & 1986, 1999       \\
SFG       & 1903, 1973       \\
STL       & 1937             \\
TBD       & 2007             \\
TEX       & 1973, 1990       \\
WSN       & 2011             \\
\end{tabular}
\caption{Mean change points for teams}
\label{tab:mean:team}
\end{table}

 

\begin{table}
\begin{tabular}{|r|r|}
Franchise & Change Point Year(s) \\
\hline
ATL       &1887             \\
BOS       &2008             \\
CHW       &1969             \\
CLE       &1966, 1981, 1993 \\
DET       &1942             \\
HOU       &2010             \\
NYM       &1974             \\
PIT       &1902, 2001       \\
SDP       &1977             \\
SFG       &1918             \\
\end{tabular}
\caption{Variance change points for teams}
\label{tab:var:team}
\end{table}

Table \ref{tab:mean:team} says some shit.  
Table \ref{tab:var:team} says some shit. 


\begin{table}
\begin{tabular}{|r|r|}
Statistic & Change Point Year(s) \\
\hline
attpergame &1945, 1975, 1992       \\
bbpergame  &1927                   \\
hrpergame  &1920, 1946, 1967, 1993 \\
rpergame   &1919, 1940             \\
sbpergame  &1919                   \\
sopergame  &1956, 1993, 2009       \\
\end{tabular}
\caption{Mean Change points for statistics}
\label{tab:mean:stat}
\end{table}

\begin{table}
\begin{tabular}{|r|r|}
Statistic & Change Point Year(s) \\
\hline
attpergame &1944, 2008             \\
hrpergame  &1927, 1944, 1995, 2007 \\
rpergame   &1935                   \\
sbpergame  &1919, 1966             \\
sopergame  &1990                   \\
\end{tabular}
\caption{Variance Change points for statistics}
\label{tab:var:stat}
\end{table}



\begin{Schunk}
\begin{Soutput}
[1] "g"          "r"          "ab"         "sb"         "hr"        
[6] "bb"         "so"         "attendance"
\end{Soutput}
\begin{Soutput}
   year_id  franch_id          g          r         ab         sb         hr 
         0          0          0          0          0          0          0 
        bb         so attendance 
         0         16          0 
\end{Soutput}
\begin{Soutput}
 iter imp variable
  1   1  bb  so  sb  attendance
  1   2  bb  so  sb  attendance
  1   3  bb  so  sb  attendance
  1   4  bb  so  sb  attendance
  1   5  bb  so  sb  attendance
  2   1  bb  so  sb  attendance
  2   2  bb  so  sb  attendance
  2   3  bb  so  sb  attendance
  2   4  bb  so  sb  attendance
  2   5  bb  so  sb  attendance
  3   1  bb  so  sb  attendance
  3   2  bb  so  sb  attendance
  3   3  bb  so  sb  attendance
  3   4  bb  so  sb  attendance
  3   5  bb  so  sb  attendance
  4   1  bb  so  sb  attendance
  4   2  bb  so  sb  attendance
  4   3  bb  so  sb  attendance
  4   4  bb  so  sb  attendance
  4   5  bb  so  sb  attendance
  5   1  bb  so  sb  attendance
  5   2  bb  so  sb  attendance
  5   3  bb  so  sb  attendance
  5   4  bb  so  sb  attendance
  5   5  bb  so  sb  attendance
\end{Soutput}
\end{Schunk}




NYY 1918: Babe Ruth got to the Yankees in 1920.  And starting in 1919 they had exactly one losing seasons between 1919 and 1965.  
BOS 1917, 1929, 1948, 1994: 
Their last world series win in the 1900s was in 1918.  
I don't know 1929. 
1948 there was a big jump in runs?  
1994: Strike year. 

LAD: 1919, 1955

ATL 1918 1949

CHW: 1919: Blaack Sox Scandal

CHC: 1919 1940
1940: War. 

Cin: 1929 1952

CLE 1920 1939
Cleveland won the world series in 1920. Major change in offensive output.  


DET 1927 1940

BAL 1919 1942

SFG 1919 1937

OAK 1920 1936 1951

PHI 1918 1937

PIT 1920 1946

STL 1919 1941

MIN 1919 1942

Pearl Harbor was 1941.  So US was in war in 1942.  

\section{Conclusions}
Baseball researchers 


\section{Acknowledgements}
We thank Michael Lopez for suggesting we do "something with change point analysis".  

\section{Supp}

All code for reproducing the analyses in this paper is publicly available at https://github.com/menawhalen/baseball\_cpt


\begin{acknowledgement}
  Please insert acknowledgments of the assistance of colleagues or similar notes of appreciation here.
\end{acknowledgement}

\begin{funding}
  Please insert information concerning research grant support here (institution and grant number). Please provide for each funder the funder’s DOI according to https://doi.crossref.org/funderNames?mode=list.
\end{funding}

\bibliographystyle{rss}
\bibliography{references.bib}




\end{document}
